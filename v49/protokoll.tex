\input{../header.tex}
\usepackage{romannum}
\usepackage{listings}
\lstset{numbers=left, numberstyle=\tiny, numbersep=5pt}
\lstset{language=Perl}
\AtBeginDocument{\pagenumbering{arabic}}

\title{\includegraphics[scale=0.8]{../logo.jpg} \\ \vspace*{1cm} VXX \\ - xxx -}

%\title{test}
\date{Durchführung: xx.xx.2021, Abgabe: xx.xx.2021}

\begin{document}

\maketitle

\tableofcontents
\newpage

\section{Ziel}

\newpage
\section{Auswertung}
Im folgenden Abschnitt werden die gesammelten Messdaten ausgewertet und analysiert.
\subsection{Justage}
Bevor mit den eigentlichen Messungen begonnen werden kann, müssen einige Justierungen vorgenommen werden.
Die Shim-Parameter sind in \autoref{tab:shim} dargestellt und gelten für die Bestimmung
von $T_1$ und $T_2$.
\begin{table}
  \centering
  \caption{Verwendetete Shim-Parameter.}
  \label{tab:shim}
  \begin{tabular}{c c c c}
    \toprule
    x & y & z & $\symup{z^2}$ \\
    \midrule
    -1.0 & -4.2 & +4.1 & -2.24 \\
    \bottomrule
  \end{tabular}
\end{table}
\FloatBarrier
Der Imaginärteil der Schwingung verschwindet für eine eingestellte Phase von $\phi=-64°$.
Für eine Frequenz von $f=\SI{21,7 }{\mega\hertz}$, weisen beide Signale Real- und Imaginärteil
keine Oszillationen mehr auf.


\subsection{Bestimmung der Spin-Gitter Relaxationszeit T1 }
Für die Messung der Spin-Gitter Relaxationszeit $T_1$ wird die $90°$
Pulslänge ($\tau_\text{90}$) auf $\SI{2,5}{\micro\second}$ und die $180°$ Pulslänge ($\tau_\text{180}$) auf
$\SI{5,08}{\micro\second}$ gestellt.
Die Messwerte sind in \autoref{tab:t1}  und der Plot in \autoref{fig:T1}zu finden.
Mit Hilfe der Funktion
\begin{equation*}
  M\left(\tau\right) = M_0 \left(1-2\exp(-\frac{\tau}{T_1})\right)
\end{equation*}
findet eine Ausgleichsrechnung statt.
Der Fit liefert die Fitparameter 
\begin{align*}
    M_0 &= \SI{1086 \pm 7}{\mV}\\% \label{eqn:M0_T1} \\
    T_1 &= \SI{2,565 \pm 0,034 }{\second}.% \label{eqn:T1}.
\end{align*}
\input{build/tabT1.tex}
\begin{figure}
    \centering
    \includegraphics[width=0.7\linewidth]{build/T1.pdf}
    \caption{Plot für die Bestimmung der Spin-Gitter Relaxationszeit.}
    \label{fig:T1}
\end{figure}
\FloatBarrier
\subsection{Bestimmung der Spin-Spin Relaxationszeit T2}
\subsubsection{Die Meiboom-Gill-Methode}
In diesem Versuchsteil wird mittels der Meiboom-Gill-Methode 
die Spin-Gitter Relaxationszeit $T_2$ bestimmt.
Der Graph welcher mit der Meiboom-Gill-Methode entstanden ist, ist in \autoref{fig:mei}
zu sehen.
\begin{figure}
  \centering
  \includegraphics[width=0.7\linewidth]{build/MG.pdf}
  \caption{Signalverlauf unter der Anwendung der Meiboom-Gill-Methode.}
  \label{fig:mei}
\end{figure}
Um im nächsten Schritt die die Relaxationszeit $T_2$ zu bestimmen werden die 
einzelnen Peaks in \autoref{fig:mei} in einen neuen Graphen in Abhängigkeit der Zeit aufgetragen.
Die dieser Graph ist in \autoref{fig:mei_fit} zu sehen.
Mit Hilfe der Funktion
\begin{equation*}
  M\left(t\right) = M_0 \exp(-\frac{t}{T_2})
\end{equation*} 
wird der Graph gefittet.
Der Fit liefert die Fitparameter 
\begin{align*}
    M_0 &= \SI{1025 \pm 5}{\mV}\\ % \label{eqn:M0_T2} \\
    T_2 &= \SI{1,696 \pm 0,019 }{\second}.% \label{eqn:T2}.
\end{align*}
\begin{figure}
  \centering
  \includegraphics[width=0.7\linewidth]{build/peaks.pdf}
  \caption{Darstellung der einzelnen Peak der Meiboom-Gill-Methode.}
  \label{fig:mei_fit}
\end{figure}
\FloatBarrier
\subsubsection{Die Carr-Purcell-Methode}
Eine andere Methode um die Relaxationszeit $T_2$ zu bestimmen, ist die Carr-Purcell-Methode.
Allerdings muss bei dieser darauf geachtet werden, dass die Pulslänge des 180° Pulses exakt
eingestellt wird.
Dies ist aber experimentell in unserem Versuch nicht realisierbar und würde zu großen
Abweichungen führen.
In \autoref{fig:carr} ist der Vollständigkeit, der Graph der Carr-Purcell-Methode abgebildet.
\begin{figure}
    \centering
    \includegraphics[width=0.7\linewidth]{build/carr_purcell.pdf}
    \caption{Graph der Carr-Purcell-Methode.}
    \label{fig:carr}
\end{figure} 

\subsection{Bestimmung der Diffusionskonstante}
Um die Diffusionskonstante der Probe zu bestimmen, wird als erster Schritt die Diffusionszeit
$T_D$ ermittelt.
Die aufgenommenen Messwerte sind in \autoref{tab:echo_2} dargestellt.
\input{build/tabEcho_2.tex}
Um die Diffusionszeitzu bestimmen werden die gemessenen Echohöhen $M(\tau)$ in Abhängigkeit der Pulslänge $\tau$ in \autoref{fig:echo_2}
graphisch dargestellt.
\begin{figure}
  \centering
  \includegraphics[width=0.7\linewidth]{build/echo_2.pdf}
  \caption{Gemessene Echohöhen in Abhängigkeit der Pulslänge.}
  \label{fig:echo_2}
\end{figure}
Anschließend werden die Messdaten mit 
\begin{equation*}
  M\left(\tau\right) = M_0 \exp(-\frac{2\tau}{T_2}) \exp(-\frac{\tau^3}{T_D}) + M_1
\end{equation*}
gefittet.
Der Fit liefert die Fitparameter 
\begin{align*}
    M_0 &= \SI{1045 \pm 12}{\mV}\\% \label{eqn:M0_diff} \\
    M_1 &= \SI{41 \pm 12}{\mV}\\% \label{eqn:M1_diff} \\
    T_D &= \SI{1,40(5)e3}{\milli\second^3}.% \label{eqn:diff_zeit}.
\end{align*}
\FloatBarrier
\begin{figure}
  \centering
  \includegraphics[width=0.7\linewidth]{build/echo_gradient.pdf}
  \caption{Fouriertransformation des vollständigen Echos.}
  \label{fig:Fouriertransformation}
\end{figure}
\FloatBarrier
Nachdem die Diffusionszeit $T_D$ bestimmt worden ist,
wird im nächsten Schritt der Magnetfeldgradient mit Hilfe einer Fouriertransformation 
des vollständigen Echoverlaufs abgeschätzt.
Die Fouriertransformation ist in \autoref{fig:Fouriertransformation} zu sehen.
Aus der Fouriertransformation ergibt sich ein Durchmesser der halbkreisförmigen Verlaufes
$d_f \approx \SI{19,5}{\kilo\hertz}$.
Daraus folgt mit 
\begin{equation*}
  g = \frac{2 \pi d_f}{\gamma d}
\end{equation*}
ein Magnetfeldgradient $g=\SI{0,103}{\tesla \meter^{-1}}$. 
Hierbei ist $\gamma= \SI{267,5e6}{\second^{-1}\tesla^{-1}}$\cite{gyro_wiki} das 
gyromagnetische Verhältnis für Protonen und $d=\SI{4,2}{\milli\meter}$ der Probeninnendurchmesser.
Im letzten Schritt kann die Diffusionskonstante mit
\begin{equation*}
  D = \frac{3}{2 T_D \gamma^2 g^2} 
\end{equation*}
ermittelt werden.
Die Diffusionskonstante beträgt $D= \SI{1,40(5)e-9}{\meter^2\per\second}$.

\subsection{Bestimmung des Molekülradius}
Der Molekülradius kann mit Hilfe der Stokes-Formel berechnet werden.
Es gilt 
\begin{equation}
  r=\frac{k_\text{B}T}{6 \pi\eta D}.
  \label{eqn:r}
\end{equation}
Hierbei ist $T = \SI{294.75}{\kelvin}$ die gemessene Temperatur  
und $\eta = \SI{890.2}{\micro\pascal\second}$\cite{vis} die Viskosität.
Mit \autoref{eqn:r} ergibt sich ein Molekülradius von 
\begin{equation*}
  r = \SI{1.73(6)e-10}{\meter}.
\end{equation*}

Um einen Vergleichswert zu generieren wird angenommen, dass die Moleküle 
in Wasser in der hexagonal dichtesten Kugelpackung angeordnet sind.
Die Einheitszelle besitzt in dieser Konstelation eine Füllung von ca. 74\%.
Die Dichte von Wasser bei einer Temperatur von $T = \SI{298.15}{\kelvin}$ ist 
$\rho=\SI{995}{\kilo\gram\per\metre^3}$\cite{dichte}. 
Das Molekülgewicht von Wasser beträgt $m = \frac{M_\text{mol}}{N_\text{A}} = \SI{2.99e-26}{\kilo\gram}$\cite{wasser}
Mit der Formel
\begin{equation*}
  r_\text{hcp} = \left(\frac{m 0,74}{\frac{4}{3}\pi\rho}\right)^{\frac{1}{3}}
\end{equation*}
ergibt sich ein Molekülradius von $r = \SI{1.74e-10}{\meter}$.
Die Abweichung von Theoriewert beträgt somit ca. 1\%.


\section{Diskussion}
Zu Beginn der Diskussion kann gesagt werden, dass die Änderung der Temperatur während 
des gesamten Versuches eine wesentliche Fehlerquelle ist. 
Die Änderung der Temperatur ist durch das wiederholte lüften des Raumes zustande gekommen und zum anderen 
durch den Erwärmungsvorgang des verwendeten Magneten.
Des weiteren kann nicht genau gesagt werden ob das permanente Magnetfeld durch 
die unterschiedlichen Shim-Parameter in der Probe optimal homogenisiert worden ist.\\
Die gemessenen Relaxationszeiten sind 
\begin{align*}
  T_1 &= \SI{2,565 \pm 0,034 }{\second}\\
  T_2 &= \SI{1,696 \pm 0,019 }{\second}.
\end{align*}
Klar erkennbar ist, dass $T_1 > T_2$ ist, welches mit der Theorie übereinstimmt.
Werden die errechneten Werte direkt mit den Theoriewerten verglichen\cite{diff}
\begin{align*}
  T_\text{1,lit} &= \SI{3.09(15)}{\second},\\
  T_\text{2,lit} &= \SI{1.52(9)}{\second},
\end{align*}
ergibt sich für $T_1$ eine Abweichung von $16,99\%$ und für $T_2$ eine Abweichung von $10,37\%$.

Die Diffusionskonstante hat in unserem Experiment der Wert 
\begin{equation*}
  D= \SI{1,40(5)e-9}{\meter^2\per\second}.
\end{equation*}
Verglichen mit dem Theoriewert\cite{diff} von 
\begin{equation*}
  D_\text{lit} = \SI{2.78(4)e-9}{\metre^2\per\second}
\end{equation*}
ergibt sich eine Abweichung von 49,64\%.
%Die Abweichung  warum so groß ? wegen halbkreisförmigen duch geschätzt,??

Der ermittle Molekülradius ist
\begin{equation*}
  r = \SI{1.73(6)e-10}{\meter}.
\end{equation*}
Verglichen mit dem Theorie von
\begin{equation*}
  r_\text{hcp}= \SI{1.74e-10}{\metre}.
\end{equation*}
ergibt sich wie bereits erwähnt eine Abweichung von ca. 1\%.
Der ermittelte Wert scheint also sehr gut zu passen. 
Allerdings muss gesagt werden, dass für die Berechnung des Theoriewerts die Dichte von Wasser bei einer 
Temperatur von $\SI{25}{\celsius}$ verwendeten worden ist.
Im Experiment herrschte aber zu diesem Zeitpunkt eine Temperatur von nur $\SI{21,6}{\celsius}$.
Außerdem ist zur Berechnung die hexagonal dichtesten Kugelpackung angenommen worden, 
welche den kleinsten Molekülradius liefert.
%\FloatBarrier
%\begin{figure}
%    \centering
%    \includegraphics[width=0.7\linewidth]{./figures/xxx.xxx}
%    \caption{xxx}
%    \label{fig:xxx}
%\end{figure}
%
%\begin{equation}\label{eq:xxx}    
%    \begin{split}
%        \lambda &= \SI{855}{\angstrom}\\
%        \delta_\text{ps} &= 0,6\cdot 10^{-6}\\
%        \delta_\text{si}&= 6,8\cdot 10^{-6} \\
%        n_\text{luft} &= 1 \\
%        n_\text{ps} &= 1 - \delta_\text{ps} \\
%        n_\text{si} &= 1 - \delta_\text{ps} 
%    \end{split}
%\end{equation}
\newpage
\nocite{*}
\printbibliography{}
\end{document}