\input{../header.tex}
\usepackage{romannum}
\AtBeginDocument{\pagenumbering{arabic}}

% \title{\includegraphics[scale=0.8]{../logo.jpg} \\ \vspace*{1cm} V14 \\ - Gammatomographie -}
\title{\includegraphics[scale=0.8]{../logo.jpg} \\ \vspace*{1cm} V44 \\ - Röntgenreflektometrie -}

%\title{test}
\date{Durchführung: 05.11.2021, Abgabe: XX.11.2021}

\begin{document}

\maketitle

\tableofcontents
\newpage


\section{Auswertung}
\subsection{Justierung und Kalibrierung}
Bevor die eigentlichen Messergebnisse ausgewertet werden, finden eine Reihe von Kalibrationsmessungen statt,
welche den Zweck erfüllen die Röntgenröhre optimal zu justieren und Zwischenergebnisse zu liefern.
Diese Kalibrationsmessungen werden im folgenden zuerst behandelt und ausgewertet.

\subsection{Der Detektorscan}

Die erste Messung ist die des Detektorscans welcher in \autoref{fig:detec} dargestellt ist. 
In diesem Plot ist die Detektorintensität in Abhängigkeit des Einfallwinkels zu sehen. 
Die Messung wurde über eine Bereich von $ -0,3° \text{ bis } 0,3° $ durchgeführt.
Die Messwerte wurden anschließend an einen Gausfit mit der Formel 
\begin{equation*}
    I(\alpha) = \frac{a}{\sigma\sqrt{2\pi}} \exp\left( \frac{-\left( \alpha - \alpha_0\right)^2}{2 \sigma} \right)
\end{equation*}
angepasst.
Die Fitparameter sind 
\begin{align*}
    a &= 107447.10 \pm 759.63 \\
    \alpha_{0} &= \SI{0.0107 \pm 0.0003}{\degree} \\
    \sigma &=  0.045 \pm 0.00037
\end{align*} 
Anhand des Graphen lässt sich die Halbwertsbreite sowie die maximale Detektorintensität ermitteln.
Die Halbwertsbreite beträgt $\text{FWHM} = 0.102°$ und die maximale Intensität $I_\text{max}=107447.10 \pm 759.63$.
\begin{figure}
    \centering
    \includegraphics[width=11cm]{build/detector_scan.pdf}
    \caption{Verlauf des Detektorscan. Kalibrationsmessung welche die Intensität in Abhängigkeit des Detektorwinkels zeigt. 
            Es gilt $\alpha_\text{f} = \alpha_\text{i}$. 
            Erkennbar ist der gaussförmige Verlauf der Intensität. Die Halbwertsbreite ist farblich rot markiert und beträgt $\text{FWHM} = 0.102°$.}
    \label{fig:detec}
\end{figure}
\FloatBarrier

\subsection{Der Z-Scan}

Mit Hilfe des Z-Scans lässt sich die Strahlenbreite der Röntgenröhre näherungsweise bestimmen.
Dazu werden die abfallende und ansteigende Flanke des Scans betrachtet. 
Der Z-Scan ist in \autoref{fig:z} zu sehen.
\begin{figure}
    \centering
    \includegraphics[width=11cm]{build/z_scan.pdf}
    \caption{Verlauf des Z-Scans. Anhand der z-Achse lässt sich die Position der Probe im Strahlengang der Röntgenröhre identifizieren. 
            Mit fallendem z-Wert wird die Probe immer weiter in der Strahlengang geschoben. Der grüne Kasten markiert die Strahlenbreite. 
            Die Strahlbreite beträgt $\SI{0,3}{\milli\meter}$. }
    \label{fig:z}
\end{figure}
\FloatBarrier
Die ermittelte Strahlbreite beträgt
\begin{equation*}
    d =\SI{0,3}{\milli\meter}. 
\end{equation*}
Aus der Strahlbreite und der Probendicke $D = \SI{20}{\milli\meter}$ lässt sich der Geometriewinkel errechnen.
\begin{align*}
    \alpha_g &= \arcsin(\frac{d}{D})\\
    \alpha_g &= \SI{0.86}{\degree}.
\end{align*}

\subsection{Der Rockingscan}

Anhand des Rockingscans \autoref{fig:dreieck} lässt sich der Geometriewinkel durch ablesen bestimmen.
Der Geometriewinkel beträgt
\begin{align*}
    \alpha_1 &= \SI{-0.76}{\degree}\\    
    \alpha_2 &= \SI{0.68}{\degree}\\    
    \alpha_g &= \frac{|\alpha_1|+|\alpha_2|}{2}\\
    \alpha_g &= \SI{0.72}{\degree}.    
\end{align*}
\begin{figure}
    \centering
    \includegraphics[width=11cm]{build/dreieck.pdf}
    \caption{Verlauf des Rockingscans. Der Rockingscan liefert durch ablesen einen Geometriewinkel von $\alpha_g = \SI{0.72}{\degree}$.}
    \label{fig:dreieck}
\end{figure}

\subsection{Der Reflektivitätsscans und die Dispersionsprofile}
In \autoref{fig:alles} ist die Reflektivität in Abhängigkeit des Einfallwinkels dargestellt.
Dazu wurde die Reflektivität auf das 5 fache der Intensität normiert, da die Messdauer nach den Kalibrationsmessung 5 mal so lang gewesen ist. 

Bei der Auswertung der Hauptmessung werden zu Beginn die Messwerte (schwarzer Graph \autoref{fig:alles}) um die Messwerte des
diffusen Scans (roter Graph) korrigiert. Die Messwerte des diffusen Scans werden von Messwerten abgezogen um auftretende Rückstreueffekte zu 
eliminieren. Der entstandene Graph (orange) in ebenfalls in \autoref{fig:alles} zu sehen. 

Anschließend wird mit Hilfe der \autoref{eq:TODO} der Geometriefaktor berechnet und der zuvor korrigierte Graph um den Geometriefaktor korrigiert. 
Der korrektur um den Geometriefaktor geschieht, da bei geringen Einfallwinkeln nicht die vollständige Intensität auf die Probenoberfläche trifft. 
Dieser Graph ist in \autoref{fig:alles} dunkel blau dargestellt. 
Werden die Kiesing Oszillationen betrachtet und die Positionen der Minima eingezeichnet, lässt sich mit der \autoref{eq:TODO} die 
Schichtdicke von Polystyrol berechnen.
Die ermittelte Schichtdicke ist
\begin{align*}
    d_\text{ps} = \SI{8,90}{\nano\meter}.
\end{align*}


\begin{figure}
    \centering
    \includegraphics[width=12cm]{build/messwerte_relativ.pdf}
    \caption{Reflektivitätsscans und Dispersionsprofile. Farblich dargestellt sind die unterschiedlichen Graphen und kritischen Winkel.
            Der hellblaue und der lilafarbene Graph sind mit dem Parratt-Algorithmus entstanden und repräsentieren eine glatte und eine raue Oberfläche der Probe. }
    \label{fig:alles}
\end{figure}
\FloatBarrier

\nocite{*}
\printbibliography{}
\end{document}