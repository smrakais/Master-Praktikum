\input{../header.tex}
\usepackage{romannum}
\AtBeginDocument{\pagenumbering{arabic}}

% \title{\includegraphics[scale=0.8]{../logo.jpg} \\ \vspace*{1cm} V14 \\ - Gammatomographie -}
\title{\includegraphics[scale=0.8]{../logo.jpg} \\ \vspace*{1cm} V44 \\ - Röntgenreflektometrie -}

%\title{test}
\date{Durchführung: 05.11.2021, Abgabe: XX.11.2021}

\begin{document}

\maketitle

\tableofcontents
\newpage


\section{theorie}

theorie\cite{anleitung}

\section{Auswertung}
\subsection{Justierung und Kalibrierung}
Bevor die eigentlichen Messergebnisse ausgewertet werden, finden eine Reihe von Kalibrationsmessungen statt,
welche den Zweck erfüllen die Röntgenröhre optimal zu justieren und Zwischenergebnisse zu liefern.
Diese Kalibrationsmessungen werden im folgenden zuerst behandelt und ausgewertet.

\subsection{Der Detektorscan}

Die erste Messung ist die des Detektorscans welcher in \autoref{fig:detec} dargestellt ist. 
In diesem Plot ist die Detektorintensität in Abhängigkeit des Einfallwinkels zu sehen. 
Die Messung wurde über eine Bereich von $ -0.3° \text{bis } 0.3° $ durchgeführt.
Die Messwerte wurden anschließend an einen Gausfit mit der Formel 
\begin{equation*}
    I(\alpha) = \frac{a}{\sigma\sqrt{2\pi}} \exp\left( \frac{-\left( \alpha - \alpha_0\right)^2}{2 \sigma} \right)
\end{equation*}
angepasst.
Die Fitparameter sind 
\begin{align*}
    a &= 107447.09636184348 +- 759.6263288761984 \\
    \alpha_{0} &= \SI{0.0107 \pm 0.0003}{\degree} \\%\0.01075214096866885 +- 0.0003669073698664512 \\
    \sigma &=  0.04494498531042028 +- 0.00036690737792524343
\end{align*} 


\begin{figure}
    \centering
    \includegraphics[width=11cm]{build/detector_scan.pdf}
    \caption{Detektorscan. Kalibrationsmessung welche die Intensität in Abhängigkeit des Detektorwinkels zeigt. Es gilt $\alpha_\text{f} = \alpha_\text{i}$. 
       Erkennbar ist der gaussförmige Verlauf der Intensität. Die Halbwertsbreite ist farblich rot markiert und beträgt $\text{FWHM} = 0.102°$.}
    \label{fig:detec}
\end{figure}
\FloatBarrier



\nocite{*}
\printbibliography{}
\end{document}