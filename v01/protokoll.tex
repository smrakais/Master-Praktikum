\input{../header.tex}
\usepackage{romannum}
\usepackage{listings}
\lstset{numbers=left, numberstyle=\tiny, numbersep=5pt}
\lstset{language=Perl}
\AtBeginDocument{\pagenumbering{arabic}}

\title{\includegraphics[scale=0.8]{../logo.jpg} \\ \vspace*{1cm} VXX \\ - xxx -}

%\title{test}
\date{Durchführung: xx.xx.2021, Abgabe: xx.xx.2021}

\begin{document}

\maketitle

\tableofcontents
\newpage

\section{Ziel}


\section{Auswertung}
Im folgenden Kapitel werden die gesammelten Messdaten sowie die Justierung der Koinzidenz und 
die Kalibrierung des Multichannel Analyzers ausgewertet.


\subsection{Justierung der Koinzidenz}
Um während der eigentlichen Messung in einer möglichst großen Ereignisrate zu messen, wird
für verschiedene Verzögerungsdauern die Zählrate am Ende der Koinzidenz betrachtet.
Dazu wird jeweils eine der Verzögerungleitungen in ihren Wert verändert und die resultierende Zählrate pro 10 Sekunden
wird aufgenommen.
Aus diesen Messwerten lässt sich anschließend ein Graph \autoref{fig:koinzidenz} plotten, 
welcher die Ereignisraten den zugehörigen Verzögerungen zuordnet.
Die Halbwertsbreite ergibt sich zu $\text{FWHM} = \SI{58}{\nano\second}$.
Das Plateau gibt die maximale Ereignisrate an. 
Diese liegt in unserem Experiment bei (gemittelt) $\SI{112,41}{\frac{1}{10 \second}}$ .
\begin{figure}
    \centering
    \includegraphics[width=0.7\linewidth]{build_new/plateau.pdf}
    \caption{Gemessene Signalrate in Abhängigkeit der Verzögerungen.}
    \label{fig:koinzidenz}
\end{figure}
\FloatBarrier
Die dazugehörigen Messwerte sind in \autoref{tab:Koinzidenz} zu finden.
%\input{build/tabKoinzidenz.tex} %OLD
\input{build_new/tabKoinzidenz_new.tex} %FELIX
\input{build_new/tabKalibration_new.tex} %FELIX
\FloatBarrier

\subsection{Kalibrierung des Multichannel Analyzers}
Um den MCA\footnote{Im folgenden wird anstelle von Multichannel Analyzer, MCA verwendet.} zu kalibrieren, werden
mit Hilfe eines Doppelimpulsgeneratiors Impulse mit unterschiedlichen Längen auf dem MCA geschickt und von diesem verarbeitet.
Dadurch kann am Computer jeweils pro Impuls der jeweilige Kanal abgelesen werden.
Mit Hilfe dieser Wertepaare vgl.\autoref{tab:Kalibration} kann im folgenden eine lineare Regression der Werte erfolgen und somit 
kann jedem Kanal des MCA eine Zeit zugeordnet werden. 
Im \autoref{fig:kalibration} ist die lineare Regression der Messwert zu sehen.
\begin{figure}
    \centering
    \includegraphics[width=0.7\linewidth]{build_new/kal.pdf}
    \caption{Graphische Darstellung der linearen Regression.}
    \label{fig:kalibration}
\end{figure}
\FloatBarrier
Die Regression liefert die Werte
\begin{equation}\label{eq:Regression}    
    \begin{split}
        m &= \num{0,04515(8)}\frac{\si{\micro\second}}{\text{Kanal}} \\
        b &= \SI{-0,041(10)}{\micro\second} \, .
    \end{split}
\end{equation}
\subsection{Bestimmung des Untergrunds}
Um den Untergrund zu bestimmen muss zunächst berechnet 
werden, wie viele Myonen im Mittel den Tank durchquert 
haben. 
\begin{equation}
  \overline{N} = \frac{N_{\symup{start}}}{T_{\symup{gesamt}}} \, .
\end{equation}
$N_\text{Start}$ steht hierbei für die Myonen, welche einen
Startimpuls ausgelöst haben.
Während der Suchzeit $T_{s}$ tun dies im Mittel
$n= \overline{N} \cdot T_{s}$ Myonen.
Anschließend wird mit Hilfe der Poissonverteilung die Anzahl
der Fehlmessungen bestimmt also genau der Fall betrachtet falls zwei 
Myonen unmittelbar hintereinander den Detektor passieren 
sollten. Es folgt mit 
$N_{\symup{start}} = \num{3.0619(17)e6}$, $T_S = \SI{20}{\micro\second}$ und $T_{\symup{gesamt}}
= \SI{147182}{\second}$.
\begin{equation}
    N_{\symup{fehl}} = \overline{N} \cdot T_S \cdot \symup e^{- \overline{N} \cdot T_S}
    \cdot N_{\symup{start}} = \num{1273.4(15)} \, .
\end{equation}
Da sich dieser Fehler auf alle Kanäle gleichermaßen 
auswirkt, ergibt sich eine Untergrundrate von 
\begin{equation}
    U = \frac{N_{\symup{fehl}}}{\text{Anzahl Kanäle}} = \num{2.8298(32)} \, 
\end{equation}
mit 450 befüllten Kanälen.

\subsection{Bestimmung der Lebensdauer}
Im letzten Abschnitt wird die Lebensdauer der Myonen bestimmt.
Durch die Kalibration des MCA lässt sich jedem Kanal eine Zeit zuordnen, sodass 
mit den unterschiedlichen Ereignisraten der Graph in \autoref{fig:fit} ensteht.
Die Exponentalfunktion mit der die Messdaten gefittet worden sind lautet
\begin{equation}
    f(t) = N_0 \cdot \symup e^{-\lambda \, t} + U_{\symup{fit}} \, .
    \label{fit}
\end{equation}
Hierbei steht $U_\text{fit}$ für die Untergrundrate.
Die Parameter des Fits sind
\begin{align}
    N_0 &= \num{242.5(16)} \ \text{pro Kanal} \\
    \lambda &= \SI{0.451(8)}{\per\micro\second} \label{dauer} \\
    U_{\symup{fit}} &= \num{1.8(13)} \ \text{pro Kanal} \label{Untergrund} \, .
\end{align}
Der inverse Abklingkoeffizient liefert schließlich die Lebensdauer des Myons.q
Diese ist 
\begin{equation}
    \tau = \SI{2.22(4)}{\micro\second} \, .
\end{equation}
\begin{figure}
    \centering
    \includegraphics[width=0.7\linewidth]{build_new/fit.pdf}
    \caption{Gefittete Ereignisrate in Abhängigkeit der Zeit.}
    \label{fig:fit}
\end{figure}

\nocite{*}
\printbibliography{}
\end{document}
%\begin{figure}
%    \centering
%    \includegraphics[width=0.7\linewidth]{./figures/xxx.xxx}
%    \caption{xxx}
%    \label{fig:xxx}
%\end{figure}
%
%\begin{equation}\label{eq:xxx}    
%    \begin{split}
%        \lambda &= \SI{855}{\angstrom}\\
%        \delta_\text{ps} &= 0,6\cdot 10^{-6}\\
%        \delta_\text{si}&= 6,8\cdot 10^{-6} \\
%        n_\text{luft} &= 1 \\
%        n_\text{ps} &= 1 - \delta_\text{ps} \\
%        n_\text{si} &= 1 - \delta_\text{ps} 
%    \end{split}
%\end{equation} 