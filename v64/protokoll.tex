\input{../header.tex}
\usepackage{romannum}
\usepackage{listings}
\lstset{numbers=left, numberstyle=\tiny, numbersep=5pt}
\lstset{language=Perl}
\AtBeginDocument{\pagenumbering{arabic}}

\title{\includegraphics[scale=0.8]{../logo.jpg} \\ \vspace*{1cm} VXX \\ - xxx -}

%\title{test}
\date{Durchführung: xx.xx.2021, Abgabe: xx.xx.2021}

\begin{document}

\maketitle

\tableofcontents
\newpage

\section{Ziel}


\section{Auswertung}
Im folgenden Abschnitt werden die erhaltenen Messwerte ausgewertet und die Ergebnisse diskutiert.
\subsection{Bestimmung des maximalen Kontrasts}
Um bei den eigentlichen Interferenzmessungen das bestmögliche Ergebnis zu erzielen ist die
Einstellung des optimalen Kontrasts notwendig.
Der Kontrast definiert als 
\begin{equation}
    K = \frac{U_\text{Max}-U_\text{Min}}{U_\text{Max}+U_\text{Min}}.
\end{equation}
Hierbei steht $U$ für die jeweilig gemessene Maximal- und Minimalspannung.
Um den größtmöglichen Kontrast zu bestimmten, wird der erste Polfilter über einen Bereich von 0 bis 180° 
variiert und mit Hilfe der drehbaren Schraube am Glasbehälter eine maximale und eine minimale Spannung am 
Multimeter eingestellt. 
Die aufgenommenen Messwerte und sich ergebenen Kontraste sind in \autoref{tab:Kontrast} dargestellt.
\input{build/tabKontrast.tex} 
\FloatBarrier
Mit Hilfe der Messwerte lässt sich eine Zusammenhang zwischen dem Kontrast und den Drehwinkel graphisch darstellen.
Dieser Plot ist in \autoref{fig:kontrast} dargestellt.
\begin{figure}
    \centering
    \includegraphics[width=0.7\linewidth]{build/Kontrast.pdf}
    \caption{Winkelabhängikeit des Kontrasts.}
    \label{fig:kontrast}
\end{figure}
\FloatBarrier
Der Kontrast hängt mit dem Polarisationswinkel $\theta$  über
\begin{equation}
    K(\theta) = \abs{ \alpha \sin{( \beta \theta
    + \gamma)}} + \delta
    \label{kontrast}
\end{equation}
zusammen. 
Die Fitparameter haben die Werte
\begin{align}
    \begin{split}
      \alpha &= \num{-0.89(6)}\\
      \beta  &= \num{2.008(29)}\\
      \gamma &= \SI{-0.18(5)}{\degree} \\
      \delta &= \num{0.020(30)}
    \end{split}
    \label{fitparameter}
\end{align}

\section{Bestimmung des Brechungsindexes von Glas}
Der Brechungsindex von Glas wird mit Hilfe der Differenzspannungsmethode bestimmt.
Dazu trifft der Laserstrahl auf einen um 45° gedrehten PBSC und anschließend in 2 Dioden.
In dem PBSC wird die Polarisatonsrichtung der überlappenden Laserstrahlen um 45° gedreht sodass
beide Laserstrahlen entweder p- oder s-polarisiert sind.
\footnote{Es spielt keine Rolle ob beide p-polarisiert oder beide s-polarisiert sind, die Hauptsache ist das sie eine 
einheitliche Polarisation aufweisen.}
Das detektierte Signal wird mit einer Operationsverärkerschaltung verarbeitet und auf einem
Oszilloskop abgebildet.
Mit der Differenzspannungsmethode ist es möglich ein wesentlich rauschärmeres Signal zu bekommen.

Der Brechungsindex lässt sich bestimmen indem der Rotationswinkel an der Stellschraube des Glasbehälters
über einen Bereich von 10° verändert wird und die dabei auftretenden Maxima/Nulldurchgänge am
Oszilloskop gezählt werden.
In \autoref{tab:Maxima} sind die Anzahlen der aufgetretenden Maxima dargestellt.
\input{build/tabMaxima_Glas.tex}
\FloatBarrier
Im Mittel sind demnach 34,7 Maxima aufgetreten.\\
Mit der Formel
\begin{equation}
    n = \left( 1 - \frac{\lambda{M}}{2T\theta_0\theta} \right)^{-1}
  \label{brech_glas}
\end{equation}
ergibt sich ein Brechungsindex von $1,566 \pm 0,021$.\\
Hierbei ist $T=\SI{1}{\milli\meter}$ die Dicke des Glases, $\lambda=\SI{632,990}{\nano\meter}$ die 
Wellenlänge des Lasers, $M$ die Anzahl der Maxima, $\theta_0$ der Verkippungswinkel der Gläser im Glashalter und
$\theta$ der Drehwinkel über einen Bereich von 10°.

%\begin{figure}
%    \centering
%    \includegraphics[width=0.7\linewidth]{./figures/xxx.xxx}
%    \caption{xxx}
%    \label{fig:xxx}
%\end{figure}
%
%\begin{equation}\label{eq:xxx}    
%    \begin{split}
%        \lambda &= \SI{855}{\angstrom}\\
%        \delta_\text{ps} &= 0,6\cdot 10^{-6}\\
%        \delta_\text{si}&= 6,8\cdot 10^{-6} \\
%        n_\text{luft} &= 1 \\
%        n_\text{ps} &= 1 - \delta_\text{ps} \\
%        n_\text{si} &= 1 - \delta_\text{ps} 
%    \end{split}
%\end{equation}
 
\nocite{*}
\printbibliography{}
\end{document}