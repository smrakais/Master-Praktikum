\input{../header.tex}
\usepackage{romannum}
\usepackage{listings}
\lstset{numbers=left, numberstyle=\tiny, numbersep=5pt}
\lstset{language=Perl}
\AtBeginDocument{\pagenumbering{arabic}}

\title{\includegraphics[scale=0.8]{../logo.jpg} \\ \vspace*{1cm} VXX \\ - xxx -}

%\title{test}
\date{Durchführung: xx.xx.2021, Abgabe: xx.xx.2021}

\begin{document}

\maketitle

\tableofcontents
\newpage

\section{Ziel}


\section{Auswertung}
Im folgenden Abschnitt werden die erhaltenen Messwerte ausgewertet und die Ergebnisse diskutiert.
\subsection{Bestimmung des maximalen Kontrasts}
Um bei den eigentlichen Interferenzmessungen das bestmögliche Ergebnis zu erzielen ist die
Einstellung des optimalen Kontrasts notwendig.
Der Kontrast definiert als 
\begin{equation}
    K = \frac{U_\text{Max}-U_\text{Min}}{U_\text{Max}+U_\text{Min}}.
\end{equation}
Hierbei steht $U$ für die jeweilig gemessene Maximal- und Minimalspannung.
Um den größtmöglichen Kontrast zu bestimmten, wird der erste Polfilter über einen Bereich von 0 bis 180° 
variiert und mit Hilfe der drehbaren Schraube am Glasbehälter eine maximale und eine minimale Spannung am 
Multimeter eingestellt. 
Die aufgenommenen Messwerte und sich ergebenen Kontraste sind in \autoref{tab:Kontrast} dargestellt.
\input{build/tabKontrast.tex} 
\FloatBarrier
Mit Hilfe der Messwert lässt sich eine Zusammenhang zwischen dem Kontrast und den Drehwinkel graphisch darstellen.
Dieser Plot ist in \autoref{fig:kontrast} dargestellt
\begin{figure}
    \centering
    \includegraphics[width=0.7\linewidth]{build/Kontrast.pdf}
    \caption{Winkelabhängikeit des Kontrasts.}
    \label{fig:kontrast}
\end{figure}


%\begin{figure}
%    \centering
%    \includegraphics[width=0.7\linewidth]{./figures/xxx.xxx}
%    \caption{xxx}
%    \label{fig:xxx}
%\end{figure}
%
%\begin{equation}\label{eq:xxx}    
%    \begin{split}
%        \lambda &= \SI{855}{\angstrom}\\
%        \delta_\text{ps} &= 0,6\cdot 10^{-6}\\
%        \delta_\text{si}&= 6,8\cdot 10^{-6} \\
%        n_\text{luft} &= 1 \\
%        n_\text{ps} &= 1 - \delta_\text{ps} \\
%        n_\text{si} &= 1 - \delta_\text{ps} 
%    \end{split}
%\end{equation}
 
\nocite{*}
\printbibliography{}
\end{document}