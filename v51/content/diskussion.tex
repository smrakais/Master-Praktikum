\newpage
\section{Diskussion}
Im folgenden Abschnitt werden die ausgewerteten Messwerte und Ergebnisse diskutiert.

\subsection{Der invertierende Verstärker / Linearverstärker}
Werden die aus den Messdaten erhaltenen Graphen (vgl. \autoref{fig:Linearverstaerker})
betrachtet, so bestätigt sie die vorangestellte Theorie und sind somit als 
wiedererkennbar zu charakterisieren. 

Auffällig ist hierbei, dass für niedrige Verstärkungsfaktoren eine größere Bandbreite erkennbar ist.
Das heißt konkret, dass bei niedrigen Verstärkungsfaktoren die Verstärkung über einen
größeren Frequenzbereich konstant gehalten werden kann.

Wird speziell der Graph in \autoref{fig:c} betrachtet, so fällt auf, dass die 
maximal gemessene Ausgangsspannung bei $\SI{50}{\volt}$ (vgl. \autoref{tab:Linearverstärker_1000 }) liegt. 
Dies ist aber nach der Theorie unmöglich, da die maximale Ausgangsspannung höchstens so groß wie 
die angeschlossene Versorgungsspannung sein kann. 
Diese beträgt bei dieser Messung aber $\SI{15}{\volt}$. 
Auch nach einiger Überlegung kann nicht beantwortet werden, warum ein so hoher Wert gemessen worden ist.

\subsection{Verhalten der Phasendifferenz bei steigender Frequenz}
Wird die Phasendifferenz bei steigernder Frequenz betrachtet, so lässt sich erkennen, das diese sinkt. 
Dieses Ergebnis ist ebenfalls mit der Theorie vereinbar.
Die Phasendifferenz sinkt bei niedrigen Verstärkungsfaktoren langsamer als bei hohen.

\subsection{Der invertierende Integrator}
Die im Experiment gesuchte Antiproportionalität zwischen der Ausgangsspannung $U_\text{A}$ und der Frequenz
$f$ kann durch \autoref{fig:integrator} bestätigt werden.
Allerdings ist im Bereich von 1000\,Hz bis 100.000\,Hz ein nicht eindeutig identifizierbares 
Verhalten der Ausgangsspannung gemessen worden. 
In diesem Bereich scheint die Ausgangsspannung zu steigen und danach wieder zu fallen.

Die Bilder, welche mit Hilfe des Oszilloskops aufgezeichnet worden, bilden 
die Erwartungen ab.

\subsection{Der invertierende Differenzierer}
Die im Experiment gesuchte Proportionalität zwischen der Ausgangsspannung $U_\text{A}$ und der Frequenz
$f$ kann hier ebenfalls durch \autoref{fig:differenzierer} bestätigt werden.
Ab einer Frequenz von ca.10.000\,Hz ist allerdings wieder ein Abfall der Ausgangsspannung gemessen worden,
ähnlich wie der Anstieg bei einem invertierenden Integrator.

\subsection{Schmitt-Trigger und Signalgenerator}
Die berechneten und gemessenen Schaltschwellen des Schmitt-Triggers stimmen
bis auf eine Abweichung von 11\% überein.
Damit kann der gemessene Wert als plausibel angenommen werden. 

Beim Signalgenerator ist hinter dem Integrator eine Dreieckspannung gemessen worden,
was ganz den Erwartungen entspricht.
Die zuvor berechnete Ausgangsfrequenz weicht von der gemessenen um 14\% ab.