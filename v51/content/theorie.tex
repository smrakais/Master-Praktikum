\section{Ziel}
Das Ziel dieses Versuchs ist es, ein Verständnis für die Funktionsweise von Operationsverstärkern zu entwickeln, indem grundlegende Schaltungen aufgebaut werden. Es sollen einige Anwendungen verwirklicht werden und so Unterschiede zwischen realen und idealen Operationsverstärkern bestimmt werden. %noch umformulieren


\section{Theorie}

\subsection{Eigenschaften eines idealen Operationsverstärkers}
Ein Operationsverstärker ist ein gleichstromgekoppelter Differenzverstärker, dessen Ausgangsspannung proportional zur Differenz der beiden Eingangsspannungen ist. Es gibt einen nicht-invertierenden und einen invertierenden Eingang.


\subsection{Unterschied zum realen Operationsverstärker}

\subsection{Schaltungen}
\subsubsection{Linearverstärker}

\subsubsection{Umkehr-Integrator}

\subsubsection{Umkehr-Differentiator}

\subsubsection{Schmitt-Trigger}




%Formeln aus der Kurzanleitung
\begin{equation*}
    \nu_{Dreieck} = \frac{R_2}{4 C R_1 R_3}
    \label{eq:Frequenz}
\end{equation*}


\begin{align}
    T &= 2 \pi R C \\ %Schwingungsdauer
    \tau &= \frac{20 R C}{|\eta|} %Abklingdauer
    \label{eq:SchwingAbkling}
\end{align}
Hier ist $\tau$ die Zeit für die Abnahme der Amplitude bis auf den e-ten Teil ihres Anfangswerts bzw. für die Zunahme der Amplitude auf das e-fache ihres Anfangswerts und $\eta$ die Dämpfung ($\eta < 0$) bzw. Enddämpfung ($\eta > 0$). Diese gibt den Bruchteil der Ausgangsspannung $U_A$ an, welcher auf den Eingang des OPV2 gegeben wird. %was ist das?